\newpage
\section*{ABREVIATURAS}
\addcontentsline{toc}{section}{ABREVIATURAS}
\begin{tabular}{p{2cm} p{12cm}}
  \si{\angstrom} & Angstrom (1 \si{\angstrom} = 10\textsuperscript{-10} \si{\meter}) \\
  AUC & \textit{Area Under Curve}, Área Bajo la Curva. Una de las estadisticas entregadas por el analisis por curva ROC de un clasificador. Valores entre 0.5 (clasificador inútil) y 1.0 (perfecto) \\
  \Ca\ & Carbono alfa. \\
  RMSD & \textit{Root Mean Square Deviation}, Raiz de la desviación cuadrada media.\\
  GDT  & \textit{Global Distance Test} Medida de la similitud entre dos estructuras con estructura secundaria identica pero con distinta estructura terciaria. Se calcula contando la cantidad de átomos a cierto corte de distancia de la estructura original.\\
  SASA & \textit{Solvent Accessible Surface Area}, Superfície Accessible al Solvente \\
  BSASA & \textit{Buried Solvent Accessible Surface Area}, Superfície Accesible al Solvente Enterrada \\
  PDB & \textit{Protein Data Bank}, Sitio web donde son publicadadas  estructuras moleculares de libre acceso. Tambien puede significar el archivo con la estructura en si. \\
\end{tabular}

