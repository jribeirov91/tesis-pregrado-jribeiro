\newpage
\section*{MÉTODOS}
\addcontentsline{toc}{section}{MÉTODOS}
\par\refstepcounter{section}
\subsection{Campos de fuerza basados en conocimiento}
\par
Los potenciales de fuerza media utilizados y derivados en este trabajo parte del supuesto de que las
fuerzas encontradas en sistemas moleculares grandes son excesivamente complejas, por lo tanto la 
única fuente de información confiable son estructuras resueltas en su estado nativo y en equilibrio. 
Si la extracción de información es exitosa, el campo de fuerza será capaz de determinar correctamente 
si un motivo en una molécula es nativo o no. 
Esta es la llamada aproximación deductiva o \textit{knowledge-based} de un potencial de fuerza 
media. (\cite{Sippl1993})
\par
Un potencial de fuerza media parte de la ley inversa de Boltzmann:
\begin{equation}
E_{ijkl} = -kT\log(f_{ijkl}) + kT\log Z \label{boltz1}
\end{equation}
La función de energía E\textsubscript{ijkl} es el llamado potencial de fuerza media. 
La variable \textit{f} es la frecuencia relativa de un cierto estado al fijar las variables i, j, k, l en 
los sistemas observados en nuestra base de datos. 
\textit{Z} representa la función de partición y no puede ser calculada experimentalmente, y se le da el 
valor de 1 (\cite{Sippl1993}). 
La ecuación \eqref{boltz1}  entonces toma la forma:
\begin{equation}
E_{ijkl} = -kT\log(f_{ijkl}) \label{boltz2}
\end{equation}
\par
Pero para utilizar exitosamente la ley inversa de Boltzmann es necesario también definir un sistema 
de referencia apropiado. 
Este se obtiene promediando un set elegido de variables del sistema, como por ejemplo k y l.
Esto nos permite extraer una característica energética general de los sistemas, las cuáles también se 
definen como un potencial de energía:
\begin{equation}
E_{kl} = -kT\log(f_{kl}) \label{boltzref}
\end{equation}
\par
Con esto, ahora podemos obtener el valor neto del potencial de fuerza media:
\begin{equation}
\Delta E^{ij}_{kl} = E^{ij}_{kl} - E_{kl} = -kT\log \left( \frac{f^{ij}_{kl}}{f_{kl}} \right)
\end{equation}
\par
En el contexto de este trabajo, nuestras variables \textit{i} y \textit{j} indican el tipo de interacción
entre dos átomos (en el caso de los potenciales SASA, solo se usa la variable \textit{i}), mientras 
que \textit{k} y \textit{l} indican distancia en la secuencia de residuos y el \textit{bin} de la 
variable geométrica a analizar, que puede ser la distancia, BSASA o SASA.
Se aplica también un factor de corrección para números bajos de observaciones en la base de datos, sugerido
por \cite{Sippl1990}. 
Así, cuando en función de \textit{l} la ecuación final toma la forma:
\begin{equation}
\Delta E^{ij}_{k}(l) = RT\log \left[1 + M_{ijk}\sigma\right] - RT\log \left[ 1 + M_{ijk}\sigma \frac{f^{ij}_{k}(l)}{f_{k}(l)} \right] \label{finalboltz}
\end{equation}
Donde \textit{M\textsubscript{ijk}} corresponde al número de observaciones de interacciones del par al 
nivel de separación \textit{k}, y $\sigma$ al peso que se le da a cada observación. 
En este trabajo se utilizó $\sigma$ = 1/50. (\cite{Sippl1990,Melo1997})


\subsection{Determinación de tipos atómicos}
\par
Para los potenciales en proteínas, se utilizaron 40 tipos atómicos compartidos para los 20 aminoácidos.
Esto es debido a que existen 98 tipos atómicos no equivalentes en total, lo que resultaría en una base
de datos con muy pocos datos para cada par de interacciones (\cite{Melo1997}).
Las definiciones se pueden ver en la Tabla \ref{table:atomprotdef}.
%tabla tipos
\newpage
\cleardoublepage
%\addcontentsline{lot}{table}{Definición de tipos atómicos para proteínas}
\begin{table}[!htp]
\begin{tabular}{ p{40pt} p{380pt} }
  \hline
  Tipo atómico & Lista de átomos \\
  \hline
  1 & \Ca\ para todos los aminoácidos excepto Glicina \\
  2 & \Ca\ Glicina \\
  3 & N para todos los aminoácidos excepto Prolina \\
  4 & C para todos los aminoácidos \\
  5 & O para todos los aminoácidos \\
  6 & Ala-\Cb, Ile-\Cgii, Ile-\Cd, Leu-\Cdi, Leu-\Cdii, Thr-\Cg, Val-\Cgi, Val-\Cgii \\
  7 & Ile-\Cb, Leu-\Cg, Val-\Cb \\
  8 & Arg-\Cb, Arg-\Cg, Asn-\Cb, Asp-\Cb, Gln-\Cb, Gln-\Cg, Glu-\Cb, Glu-\Cg, His-\Cb, Ile-\Cgi, Leu-\Cb, Lys-\Cb, Lys-\Cg, Lys-\Cd, Met-\Cb, Phe-\Cb, Pro-\Cb, Pro-\Cg, Trp-\Cb, Tyr-\Cb \\
  9 & Met-S\textsubscript{\text{\textdelta}} \\
 10 & Pro-N \\
 11 & Phe-\Cg, Trp-\Cdii, Tyr-\Cg \\
 12 & Phe-\Cdi, Phe-\Cdii, Phe-\Cei, Phe\Ceii, Phe-\Cz, Trp-\Ceiii, Trp-\Cz, Trp-\Cziii, Trp-\Cetaii, Tyr-\Cdi, Tyr-\Cdii, Tyr-\Cei, Tyr-\Ceii \\
 13 & Trp-\Cg \\
 14 & Trp-\Ceii \\
 15 & Ser-\Cb \\
 16 & Ser-O\textsubscript{\text{\textgamma}}, Thr-O\textsubscript{\text{\textgamma}} \\
 17 & Thr-\Cb \\
 18 & Asn-N\textsubscript{\text{\textdelta}2}, Gln-N\textsubscript{\text{\textepsilon}2} \\
 19 & Cys-S\textsubscript{\text{\textgamma}} \\
 20 & Lys-N\textsubscript{\text{\textzeta}} \\
 21 & Arg-\Cz \\
 22 & Arg-N\textsubscript{\text{\texteta}1}, Arg-N\textsubscript{\text{\texteta}2} \\
 23 & His-\Cg \\
 24 & His-\Cdii, Trp-\Cdi \\
 25 & His-N\textsubscript{\text{\textepsilon}2} \\
 26 & His-\Cei \\
 27 & Asp-\Cg, Glu-\Cd \\
 28 & Asp-O\textsubscript{\text{\textdelta}1}, Asp-O\textsubscript{\text{\textdelta}2}, Glu-O\textsubscript{\text{\textepsilon}1}, Glu-O\textsubscript{\text{\textepsilon}2} \\
 29 & Cys-\Cb, Met-\Cg \\
 30 & Met-\Ce \\
 31 & Tyr-\Cz \\
 32 & Pro-\Cd \\
 33 & Asn-\Cg, Gln-\Cd \\
 34 & Asn-O\textsubscript{\text{\textdelta}1}, Gln-O\textsubscript{\text{\textepsilon}1} \\
 35 & Lys-\Ce \\
 36 & Arg-N\textsubscript{\text{\textepsilon}} \\
 37 & Arg-\Cd \\
 38 & His-N\textsubscript{\text{\textdelta}1} \\
 39 & Trp-N\textsubscript{\text{\textepsilon}1} \\
 40 & Tyr-O\textsubscript{\text{\texteta}} \\
 \hline
\end{tabular}
\caption{Definiciones de átomos pesados utilizadas para potenciales en proteínas.}
\label{table:atomprotdef}
\end{table}
%fin tabla tipos
\newpage
\clearpage
\par
En el caso de los potenciales para ADN y ARN, se utilizaron 23 tipos atómicos distintos descritos 
por \cite{Capriotti2011} para moléculas de ARN. A estos se agregaron dos tipos más, 24 y 25, 
correspondientes a los carbonos C5 y C7 (nombres IUPAC) del nucleótido timina.
Estas definiciones están en la Tabla \ref{table:atomnadef}.
%comienzo tabla nucleótidos
\newpage
\cleardoublepage
\begin{table}[!htp]
\begin{tabular}{p{40pt} p{380pt}}
  \hline \\
  Tipo atómico & Lista de átomos (nombres IUPAC) \\
  \hline \\
  1 & OP1, OP2, OP3 para todos los nucleótidos \\
  2 & P para todos los nucleótidos \\
  3 & O5' para todos los nucleótidos \\
  4 & C5' para todos los nucleótidos \\
  5 & C5', C3', C2' para todos los nucleótidos \\
  6 & O2', O3' terminales \\
  7 & C1' para todos los nucleótidos \\
  8 & O4' para todos los nucleótidos \\
  9 & N1 pirimidinas; N9 purinas \\
 10 & C8 purinas \\
 11 & N3, N7 en purinas; N1 en A; N3 en \\
 12 & C5 purinas \\
 13 & C4 purinas \\
 14 & C2 en A \\
 15 & C6 en A; C4 en C \\
 16 & N6 en A; N4 en C; N2 en G \\
 17 & C2 en G \\
 18 & C6 en G; C4 en U,T \\
 19 & O2 pirimidinas; O6 en G; O4 en U,T \\
 20 & C2 pirimidinas \\
 21 & C6 pirimidinas \\
 22 & C6 pirimidinas \\
 23 & N1 en G; N3 en U,T \\
 24 & C5 en T \\
 25 & C7 en T \\
 \hline
\end{tabular}
\caption{Definiciones de átomos pesados utilizadas para potenciales en ARN y ADN. 
Se consideran tanto nucleótidos como deoxinucleótidos.}
\label{table:atomnadef}
\end{table}
%fin tabla nucleotidos
\newpage
\clearpage
\subsection{Derivación de potenciales basados en distancias y conteos de átomos}
\par
\subsection{Cálculo de la superficie accesible al solvente de una molécula}
\par
\subsection{Cálculo de las subsuperfícies de interacción}
\par
\subsection{Derivación de potenciales basados en BSA}
\par
\subsection{Derivación de potenciales basados en SASA}
\par
\subsection{Cálculo del IP (\textit{Information Product})}
